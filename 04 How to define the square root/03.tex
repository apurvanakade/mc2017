



\documentclass{article}
\usepackage{amsmath, amsthm}
\usepackage{amssymb}
\usepackage{mathtools}
\usepackage[all,cmtip]{xy}
\usepackage{color}


\setcounter{tocdepth}{4}

\renewenvironment{proof}{ {\bfseries Proof:}}{\qed}

\newtheoremstyle{mytheorem}%                % Name
{}%                                     % Space above
{}%                                     % Space below
{\itshape}%                                     % Body font
{0pt}%\parindent}%                                     % Indent amount
{\bfseries}%                            % Theorem head font
{.}%                                    % Punctuation after theorem head
{ }%                                    % Space after theorem head, ' ', or \newline
{}%                                     % Theorem head spec (can be left empty, meaning `normal')

\theoremstyle{mytheorem}
\newtheorem{thm}{Theorem}[section]
\newtheorem{proposition}[thm]{Proposition}
\newtheorem{lemma}[thm]{Lemma}
\newtheorem{corollary}[thm]{Corollary}


\newtheoremstyle{mydefinition}%                % Name
{}%                                     % Space above
{}%                                     % Space below
{}%                                     % Body font
{0pt}%\parindent}%                                     % Indent amount
{\bfseries}%                            % Theorem head font
{.}%                                    % Punctuation after theorem head
{ }%                                    % Space after theorem head, ' ', or \newline
{}%                                     % Theorem head spec (can be left empty, meaning `normal')

\theoremstyle{mydefinition}
\newtheorem{definition}[thm]{Definition}
\newtheorem{example}[thm]{Example}
\newtheorem{exercise}[thm]{Exercise}
\newtheorem{remark}[thm]{Remark}
%\newtheorem{ques}[thm]{Q.}
\newtheorem*{ques}{Question}
%\newtheorem{ans}[thm]{Ans.}
\newtheorem*{ans}{Ans}



\numberwithin{equation}{section}

%Real numbers, complex numbers, etc.
\newcommand{\R}{\mathbb{R}}
\newcommand{\C}{\mathbb{C}}
\newcommand{\Z}{\mathbb{Z}}
\newcommand{\Q}{\mathbb{Q}}
\renewcommand{\P}{\mathbb{P}}

%How does latex not have these?
\DeclareMathOperator{\Ad}{Ad}
\DeclareMathOperator{\ad}{ad}
\DeclareMathOperator{\tr}{tr}
\DeclareMathOperator{\Tr}{Tr}
\DeclareMathOperator{\Hom}{Hom}
\DeclareMathOperator{\Spec}{Spec}
\DeclareMathOperator{\im}{im}
\DeclareMathOperator{\rank}{rank}
\DeclareMathOperator{\Exists}{\exists}
\DeclareMathOperator{\Forall}{\forall}

\DeclareMathOperator*{\colim}{colim}
\DeclareMathOperator*{\holim}{holim}
\DeclareMathOperator*{\hocolim}{hocolim}


%fractions and inner product
\newcommand{\pr}[2][\:]{\frac{\partial #1}{\partial #2}}
\newcommand{\innerp}[2]{\langle #1, #2 \rangle}

\newcommand*\conj[1]{\overline{#1}}
\newcommand*\norm[1]{\lVert #1 \rVert}

\renewcommand{\figurename}{Fig.}
\usepackage{float}
\usepackage{wrapfig}

\usepackage{enumitem}
\setlist[enumerate]{itemsep=0mm}
\usepackage{geometry}
\geometry{
	a4paper,
	total={170mm,257mm},
	left=20mm,
	top=20mm
}


\usepackage{fancyhdr}
\pagestyle{fancy}
\lhead{\scshape Apurva Nakade}
%\rhead{\scshape Mathcamp 2017}
\renewcommand*{\thepage}{\small\arabic{page}}

\DeclareMathOperator{\re}{Re}



\begin{document}
\title{Fermat's Theorem for Polynomials}
\author{Apurva Nakade}
\thispagestyle{fancy}
\maketitle


\begin{thm}[Riemann-Hurwitz]
	Given an $N$ sheeted branched covering map of compact Riemann surfaces $\pi:S' \rightarrow S$ we have the identity
	\begin{align}
		g(S') = N(g(S) - 1) + 1 + \sum \limits_{P \in S'} \dfrac{e_P - 1}{2}
	\end{align}
	where $g(S)$ and $g(S')$ denote the genus of $S$ and $S'$, the sum is over the ramified points $P$ and $e_P$ denotes the ramification degree of $P$.
\end{thm}

\begin{cor}
	\label{thm:thm1}
	If there is a branched covering $\pi: S' \rightarrow S$ of compact Riemann surfaces then $g(S') \ge g(S)$. In particular there is no branched covering $\pi: \widehat \C \rightarrow S$ unless $S = \widehat \C$.
\end{cor}
\begin{proof}
	$N$ is at least 1 and the sum $\sum_{P \in S'} \dfrac{e_P - 1}{2}$ is always non-negative, the result follows.
\end{proof}

\begin{thm}
	Every non-constant complex differentiable map between compact Riemann surfaces is a branched covering.
\end{thm}

For $d>2$ denote by $S_d$ the compactified Riemann surface defined by the equation
\begin{align}
	S_d = \mbox{ compacification of } \{ (z,w): z^d + w^d = 1\}
\end{align}
\begin{lem}
	\label{thm:thm2}
	Genus of $S_d$ is ${d-1 \choose 2}$, hence for $d>2$ the genus is at least 1.
\end{lem}
\begin{proof}
	The projection onto the $z$ coordinate and the equation $w^d = 1 - z^d $ gives us a natural $d$ sheeted branched covering $S_d \rightarrow \widehat C$. This covering is branched exactly over the roots of unity where all the $d$ sheets ramify. Further the monodromy at each branch point is exactly $\Z/d$ and there are $d$ branch points so the point at $\infty$ is not a branch point (see Figure). Thus there are $d$ many branch points with ramification degree $d$. Plugging into the Riemann-Hurwitz formula we get
	\begin{align}
		g(S_d) & = d(0 - 1) + 1 + \dfrac{d(d - 1)}{2}      \\
		       & = \dfrac{(d-1)(d-2)}{2} = {d-1 \choose 2}
	\end{align}
\end{proof}

\begin{thm}
	For $d>2$ are no non-constant solutions of the equation
	\begin{align}
		\label{eq:eq1}
		x(t)^d + y(t)^d = z(t)^d
	\end{align}
	where $x(t), y(t), z(t)$ are polynomials with complex coefficients.
\end{thm}
\begin{proof}
	Suppose $x(t)$, $ y(t)$, $z(t)$ are non-constant polynomials satisfying the equation \eqref{eq:eq1}. Without any loss of generality assume that $x(t)$, $ y(t)$, $z(t)$ have no common factors. We can define a complex differentiable map
	\begin{align}
		\pi : \widehat {\C} & \rightarrow S_d                                             \\
		p                   & \mapsto \left(\dfrac{x(p)}{z(p)}, \dfrac{y(p)}{z(p)}\right)
	\end{align}
	where $p$ is either a 0 of $z(t)$ or $p = \infty$ then $p$ maps to $\lim \limits_{p \rightarrow \infty}\left(\dfrac{x(p)}{z(p)}, \dfrac{y(p)}{z(p)}\right)$. Because the Riemann surface is compact all limits exist and the above map is well defined. As it is defined using polynomials it is also complex differentiable and hence is a branched covering. Together with Lemma \ref{thm:thm2} this contradicts Corollary \ref{thm:thm1}.
\end{proof}





\end{document}
