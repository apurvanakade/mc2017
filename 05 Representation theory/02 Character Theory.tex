



\documentclass{article}
\usepackage{amsmath, amsthm}
\usepackage{amssymb}
\usepackage{mathtools}
\usepackage[all,cmtip]{xy}
\usepackage{color}


\setcounter{tocdepth}{4}

\renewenvironment{proof}{ {\bfseries Proof:}}{\qed}

\newtheoremstyle{mytheorem}%                % Name
{}%                                     % Space above
{}%                                     % Space below
{\itshape}%                                     % Body font
{0pt}%\parindent}%                                     % Indent amount
{\bfseries}%                            % Theorem head font
{.}%                                    % Punctuation after theorem head
{ }%                                    % Space after theorem head, ' ', or \newline
{}%                                     % Theorem head spec (can be left empty, meaning `normal')

\theoremstyle{mytheorem}
\newtheorem{thm}{Theorem}[section]
\newtheorem{proposition}[thm]{Proposition}
\newtheorem{lemma}[thm]{Lemma}
\newtheorem{corollary}[thm]{Corollary}


\newtheoremstyle{mydefinition}%                % Name
{}%                                     % Space above
{}%                                     % Space below
{}%                                     % Body font
{0pt}%\parindent}%                                     % Indent amount
{\bfseries}%                            % Theorem head font
{.}%                                    % Punctuation after theorem head
{ }%                                    % Space after theorem head, ' ', or \newline
{}%                                     % Theorem head spec (can be left empty, meaning `normal')

\theoremstyle{mydefinition}
\newtheorem{definition}[thm]{Definition}
\newtheorem{example}[thm]{Example}
\newtheorem{exercise}[thm]{Exercise}
\newtheorem{remark}[thm]{Remark}
%\newtheorem{ques}[thm]{Q.}
\newtheorem*{ques}{Question}
%\newtheorem{ans}[thm]{Ans.}
\newtheorem*{ans}{Ans}



\numberwithin{equation}{section}

%Real numbers, complex numbers, etc.
\newcommand{\R}{\mathbb{R}}
\newcommand{\C}{\mathbb{C}}
\newcommand{\Z}{\mathbb{Z}}
\newcommand{\Q}{\mathbb{Q}}
\renewcommand{\P}{\mathbb{P}}

%How does latex not have these?
\DeclareMathOperator{\Ad}{Ad}
\DeclareMathOperator{\ad}{ad}
\DeclareMathOperator{\tr}{tr}
\DeclareMathOperator{\Tr}{Tr}
\DeclareMathOperator{\Hom}{Hom}
\DeclareMathOperator{\Spec}{Spec}
\DeclareMathOperator{\im}{im}
\DeclareMathOperator{\rank}{rank}
\DeclareMathOperator{\Exists}{\exists}
\DeclareMathOperator{\Forall}{\forall}

\DeclareMathOperator*{\colim}{colim}
\DeclareMathOperator*{\holim}{holim}
\DeclareMathOperator*{\hocolim}{hocolim}


%fractions and inner product
\newcommand{\pr}[2][\:]{\frac{\partial #1}{\partial #2}}
\newcommand{\innerp}[2]{\langle #1, #2 \rangle}

\newcommand*\conj[1]{\overline{#1}}
\newcommand*\norm[1]{\lVert #1 \rVert}

\renewcommand{\figurename}{Fig.}
\usepackage{float}
\usepackage{wrapfig}

\usepackage{enumitem}
\setlist[enumerate]{itemsep=0mm}
\usepackage{geometry}
\geometry{
	a4paper,
	total={170mm,257mm},
	left=20mm,
	top=20mm
}


\usepackage{fancyhdr}
\pagestyle{fancy}
\lhead{\scshape Apurva Nakade}
%\rhead{\scshape Mathcamp 2017}
\renewcommand*{\thepage}{\small\arabic{page}}

\DeclareMathOperator{\Ce}{Re}



\begin{document}
\title{Crash Course on Representation Theory - Day 2}
\author{Apurva Nakade}
\thispagestyle{fancy}
\maketitle



Character theory is the part of representation theory that allows one to determine the `prime' decomposition of representations.

\section{Character Theory}
Given a linear transformation $A \in GL(V)$ we can define the trace of $A$ to be the sum of diagonal entries of $A$ in any basis, recall that the trace does not depend upon the choice of any basis. For a representation $\rho$ we can post-compose with the trace map, the resulting function is called the \textbf{character} of the representation.
\begin{align}
	\chi_{\rho}: G \xrightarrow{\rho} GL(V) \xrightarrow{\mathrm{trace}} \C
\end{align}
The trace is a single number associated to a matrix and hence contains very little information about the matrix, but as it turns out, the character of a representation (which is a function $G \rightarrow \C$) contains enough information to differentiate between representations. We say that the character is a complete invariant of representations.


\subsection{Orthogonality relations}
We can use characters to define a hermitian inner product on the space of representations! For representations $\rho_1, \rho_2$ define an inner product and norm as
\begin{align}
	\innerp{\rho_1}{\rho_2} & = \dfrac{1}{|G|} \sum \limits _ {g \in G} {\chi_{\rho_1}(g)} \conj{\chi_{\rho_2}(g)} \\
	\norm{\rho_1}^2         & = \innerp{\rho_1}{\rho_1}
\end{align}

The following is an EXTREMELY useful theorem in character theory (which requires nothing more than linear algebra to prove).
\begin{thm}[Orthogonality relations] $\:$
	Let $\rho_1, \rho_2, \ldots, \rho_d$ be the distinct irreducible representations of $G$. If $\rho$ is an arbitrary representation which has a decomposition $\sigma = \rho_1^{k_1} \oplus \rho_2^{k_2} \oplus \cdots \oplus \rho_d^{k_d} $ then
	\begin{align}
		\innerp{\rho_i}{\sigma} & = k_i
	\end{align}
	In particular, we have $\norm{\chi_{\rho_i}} = 1$ for all $i$ and $\innerp{\chi_{\rho_1}}{\chi_{\rho_2}} = 0$ for $i \neq j$.
\end{thm}

\subsection{Character Tables}
By the cyclicity of trace we have
\begin{align*}
	\chi_\rho(h^{-1}gh) = \mathrm{trace}(\rho(h)^{-1} \rho(g) \rho(h)) = \mathrm{trace}(\rho(h)\rho(h)^{-1} \rho(g)) = \mathrm{trace}(\rho(g)) = \chi_\rho(g)
\end{align*}
and so to determine the character of a representation it is enough to determine the character of one element in each conjugacy class. Such information is typically organized in a table called the \textbf{character table}. The character tables from the examples from yesterday's notes are as follows.

\begin{align*}
	\mbox{
	\begin{tabular}{r|ccc}
	$\Z/3$      & \{0\}     & \{1\}                              & \{2\}                                \\
	\hline
	            & 1         & 1                                  & 1                                    \\
	            & 1         & $\omega$                           & $\omega^2$                           \\
	            & 1         & $\omega^2$                         & $\omega$
	\end{tabular}
	}
	&\qquad \qquad
	\mbox{
	\begin{tabular}{r|ccc}
	$S_3(=D_6)$ & $\{(1)\}$ & $\{(1 \: 2); (2 \: 3); (1 \: 3)\}$ & $\{ (1 \: 2 \: 3); (1 \: 3 \: 2) \}$ \\
	\hline
	trivial     & 1         & 1                                  & 1                                    \\
	sign        & 1         & -1                                 & 1                                    \\
	2-dim       & 1         & 0                                  & $2\cos(2 \pi / 3)$
	\end{tabular}
	}
\end{align*}

\begin{center}

\end{center}

\begin{center}

\end{center}

\begin{center}
	\begin{table}[H]
		\begin{center}
			\begin{tabular}{r|ccccc}
				$Q_8$             & $\{ 1 \}$ & $\{ -1 \}$ & $\{ i, -i \}$ & $\{ j, -j \}$ & $\{ k, -k \}$ \\	\hline
				trivial           & 1         & 1          & 1             & 1             & 1             \\
				$\mathrm{sign}_i$ & 1         & 1          & 1             & -1            & -1            \\
				$\mathrm{sign}_j$ & 1         & 1          & -1            & 1             & -1            \\
				$\mathrm{sign}_k$ & 1         & 1          & -1            & -1            & 1             \\
				2-dim             & 2         & -2         & 0             & 0             & 0
			\end{tabular}
		\end{center}
	\end{table}
\end{center}


\section{Frobenius determinant}
The Frobenius determinant can be easily computed using linear algebra. We'll do it using representation theory, this is completely unnecessary.


The first thing to observe is that the Frobenius matrix $F_n$ is constructed out of the regular representation for the group $\Z/n$. Consider the $n$ dimensional vector space with basis $e_0, e_1, \ldots, e_{n-1}$ on which $\Z/n$ acts as:
\begin{align}
	\rho(k)(e_j) & = e_{(k+j \mod n)}
\end{align}
The $j^{th}$ column of the matrix $\rho(k)$ has the 1 in the $(k+j \mod n)^{th}$ place and 0 everywhere else. This is exactly the place where the $x_k's$ occur in the Frobenius matrix!!! Hence we get the identity
\begin{align}
	\label{eq:eq1}
	F_n = \sum \limits_{k=0}^n x_k \rho(k) = \sum \limits_{k=0}^n x_k \rho(1)^k
\end{align}
For $0 \le l < n$ let $\rho_l$ be the 1 dimensional irreducible representations of $\Z/n$ with $\rho_l(1) = \omega^l$ where $\omega = e^{2 \pi i /n}$. For some constants $c_i$ we must have
\begin{align}
	\rho = \rho_0 ^ {c_0} \oplus \cdots \oplus \rho_{n-1} ^ {c_{n-1}}
\end{align}
To compute $c_l$ we compute the inner product $c_l = \innerp{\rho_l}{\rho}=\sum \limits _ {k \in \Z/n} {\chi_{\rho_l}(k)} \conj{\chi_{\rho}(k)}/n$. The matrices $\rho(k)$ have all the diagonal entries 0 except for $k = 0$ which has all the diagonal entries 1 and hence $\chi_\rho(k) = \begin{cases} n \mbox{ if k = 0} \\ 0 \mbox{ otherwise} \end{cases}$ so that $c_l = 1$ for all $l$ and we get
\begin{align}
	\rho = \rho_0 \oplus \rho_1 \oplus \cdots \oplus \rho_{n-1}
\end{align}
We can choose a basis such that the matrix $\rho(1)$ is a diagonal matrix with entries $\rho_l(1) = \omega^l$, plugging in \eqref{eq:eq1}
\begin{align}
	F_n = \sum \limits_{k=0}^n \begin{bmatrix} x_k &   &   &   \\ &  \omega^k x_k & & \\ & & \ddots & \\ & & & \omega^{(n-1)k} x_k & \end{bmatrix}
	&=
	\begin{bmatrix} \sum \limits_{k=0}^n x_k       &   &   &   \\ & \sum \limits_{k=0}^n \omega^k x_k & & \\ & & \ddots & \\ & & & \sum \limits_{k=0}^n \omega^{(n-1)k}x_k & \end{bmatrix} \\
	\implies \det F_n	= \prod \limits_{l=0}^{n-1} \sum \limits_{k=0}^n \omega^{kl} x_k &= \prod \limits_{l=0}^{n-1}\left(x_0 + \omega^{l} x_1 + \cdots + \omega^{(n-1)l}x_{n-1}\right)
\end{align}
\begin{flushright}
	$\qed$
\end{flushright}
This naturally leads to the following question.
\begin{ques}
	Consider the Frobenius matrix $F_G$ whose entries are obtained from the multiplication table for $G$ i.e. the rows and columns are indexed by the entries of $G$ and the $g,h^{th}$ entry is $x_{gh^{-1}}$. What is $\det F_G$?
\end{ques}
The above proof generalizes directly to an arbitrary group and the determinant can be computed in terms of the regular representation of $G$ and the irreducible representations of $G$.













\iffalse
\subsection{Standard representation of $S_n$}
As we saw earlier the standard representation of $S_n$ breaks down into two representations $W \oplus W^{\perp}$ where $W$ is the 1 dimensional vectors space spanned by $e_1 + \cdots + e_n$ and $W^{\perp}$ is the perpendicular vector space consisting of vectors of the form $\{ c_1 e_1 + \cdots + c_n e_n : c_1 + \cdots + c_n = 0\}$
As we saw earlier the standard representation of $S_n$ breaks down into two representations $W \oplus W^{\perp}$ where $W$ is the 1 dimensional vectors space spanned by $e_1 + \cdots + e_n$ and $W^{\perp}$ is the perpendicular vector space consisting of vectors of the form $\{ c_1 e_1 + \cdots + c_n e_n : c_1 + \cdots + c_n = 0\}$
\begin{proposition}
	\begin{align}
		\dfrac{1}{n!}\sum\limits_{\sigma \in S_n} (\mbox{number of fixed points of } \sigma)^2 = 2
	\end{align}
\end{proposition}
\fi






\end{document}
