\iffalse
It's unfair to say that mathematicians aren't real doctors, we perform surgeries all the time. In this class we'll introduce the notion of a topological manifold via simplicial (delta) complexes. Spend a day or two doing examples and go over several notions like orientation, cobordism and of course surgery.

Keywords: simplicial complex, manifold, orientation, cobordism, surgery

Type: Lecture
Homework: Recommended
Prereqs: None
\fi




\documentclass{article}
\usepackage{amsmath, amsthm}
\usepackage{amssymb}
\usepackage{mathtools}
\usepackage[all,cmtip]{xy}
\usepackage{color}


\setcounter{tocdepth}{4}

\renewenvironment{proof}{ {\bfseries Proof:}}{\qed}

\newtheoremstyle{mytheorem}%                % Name
{}%                                     % Space above
{}%                                     % Space below
{\itshape}%                                     % Body font
{0pt}%\parindent}%                                     % Indent amount
{\bfseries}%                            % Theorem head font
{.}%                                    % Punctuation after theorem head
{ }%                                    % Space after theorem head, ' ', or \newline
{}%                                     % Theorem head spec (can be left empty, meaning `normal')

\theoremstyle{mytheorem}
\newtheorem{thm}{Theorem}[section]
\newtheorem{proposition}[thm]{Proposition}
\newtheorem{lemma}[thm]{Lemma}
\newtheorem{corollary}[thm]{Corollary}


\newtheoremstyle{mydefinition}%                % Name
{}%                                     % Space above
{}%                                     % Space below
{}%                                     % Body font
{0pt}%\parindent}%                                     % Indent amount
{\bfseries}%                            % Theorem head font
{.}%                                    % Punctuation after theorem head
{ }%                                    % Space after theorem head, ' ', or \newline
{}%                                     % Theorem head spec (can be left empty, meaning `normal')

\theoremstyle{mydefinition}
\newtheorem{definition}[thm]{Definition}
\newtheorem{example}[thm]{Example}
\newtheorem{exercise}[thm]{Exercise}
\newtheorem{remark}[thm]{Remark}
%\newtheorem{ques}[thm]{Q.}
\newtheorem*{ques}{Question}
%\newtheorem{ans}[thm]{Ans.}
\newtheorem*{ans}{Ans}



\numberwithin{equation}{section}

%Real numbers, complex numbers, etc.
\newcommand{\R}{\mathbb{R}}
\newcommand{\C}{\mathbb{C}}
\newcommand{\Z}{\mathbb{Z}}
\newcommand{\Q}{\mathbb{Q}}
\renewcommand{\P}{\mathbb{P}}

%How does latex not have these?
\DeclareMathOperator{\Ad}{Ad}
\DeclareMathOperator{\ad}{ad}
\DeclareMathOperator{\tr}{tr}
\DeclareMathOperator{\Tr}{Tr}
\DeclareMathOperator{\Hom}{Hom}
\DeclareMathOperator{\Spec}{Spec}
\DeclareMathOperator{\im}{im}
\DeclareMathOperator{\rank}{rank}
\DeclareMathOperator{\Exists}{\exists}
\DeclareMathOperator{\Forall}{\forall}

\DeclareMathOperator*{\colim}{colim}
\DeclareMathOperator*{\holim}{holim}
\DeclareMathOperator*{\hocolim}{hocolim}


%fractions and inner product
\newcommand{\pr}[2][\:]{\frac{\partial #1}{\partial #2}}
\newcommand{\innerp}[2]{\langle #1, #2 \rangle}

\newcommand*\conj[1]{\overline{#1}}
\newcommand*\norm[1]{\lVert #1 \rVert}

\renewcommand{\figurename}{Fig.}
\usepackage{float}
\usepackage{wrapfig}

\usepackage{enumitem}
\setlist[enumerate]{itemsep=0mm}
\usepackage{geometry}
\geometry{
	a4paper,
	total={170mm,257mm},
	left=20mm,
	top=20mm
}


\usepackage{fancyhdr}
\pagestyle{fancy}
\lhead{\scshape Apurva Nakade}
%\rhead{\scshape Mathcamp 2017}
\renewcommand*{\thepage}{\small\arabic{page}}

\rhead{\scshape Mathcamp 2017 : All things Manifoldy}
\begin{document}
\title{Modular origami}
\author{Apurva Nakade}
\thispagestyle{fancy}
\maketitle


\section{Orientation of surfaces}
Let us start with surfaces. A surface is called orientable if has two sides. A sphere or a torus are easily seen to be orientable. In fact any surface that can be \textit{embedded} in $\R^3$ is orientable. A Mobius strip on the other hand (which is not a surface as it has a boundary) is not orientable. With a little more mental effort one might be able to see that a Klein bottle or a Projective space are not orientable. We want an equivalent definition which can be used on gluing diagrams.

We begin by defining orientation of a polygon. An \textbf{orientation of a polygon} is simply a cyclic ordering of it's vertices. For example this is one of the two possible ways to orient a triangle with vertices $(0,1,2)$.

\begin{center}
	\begin{tabular}{c}
		\centering \includegraphics[height=3cm]{../noImageAvailable}
	\end{tabular}
\end{center}

\begin{exercise}
	This is related to the fact that the polygon has two sides via the right hand rule in physics. Do you see the connection?
\end{exercise}

An \textbf{orientation} of (a gluing diagram of) a surface is a compatible choice of orientation for each triangle, where the orientations of two adjacent triangles need to be compatible in the following way,
\begin{center}
	\begin{tabular}{c c c}
		\centering \includegraphics[height=3cm]{../noImageAvailable} & \: & \centering \includegraphics[height=3cm]{../noImageAvailable}
	\end{tabular}
\end{center}
This forces an edge to be directed in two different directions in adjacent triangles.

\begin{exercise}
	Add the diagonals to the standard gluing diagrams to obtain triangulations and try to find orientations for them. Conclude that $S^2,T$ are orientable and $\R\P^2,K$ are not.
\end{exercise}

\begin{exercise}
	Use gluing diagrams to show that $T\#T$ is orientable. Generalize this to argue that $T ^{\# n}$ are orientable.
\end{exercise}

This same method also allows us to understand manifolds in higher dimensions as well.







\section{Simplices}
As we go to higher dimensions polygons (or is it polytopes?) become whacky are themselves quite hard to understand. So instead of looking at arbitrary polygons we look at the simplest polygons: triangles. We'll upgrade the definition of a triangle to a simplex which can live in arbitrary dimensions.

\begin{definition}
	An $n$ dimensional \textbf{simplex} $\Delta^n$ is any set which is homeomorphic to the following region in $\R^{n}$.
	\begin{align}
		\{ (x_0, x_1, \cdots, x_n) : x_0 + \cdots + x_n \le 1 \mbox{ and each } x_i \ge 0  \}
	\end{align}
\end{definition}
A 1 dimensional simplex is a segment and a 2 dimensional simplex is a triangle.
Simplices are topologists best friend.

Note that by adding extra diagonal lines we could have made the gluing diagrams entirely out of triangles.
We represent a simplex by its set of vertices.

\begin{figure}[h]
	\centering \includegraphics{../noImageAvailable}
	\caption{Gluing diagram with labeled simplices}
	\label{}
\end{figure}

As with origami you can glue these simplices any way you want and get relaly interesting objects. We're going to glue them to create manifolds.

\section{Manifolds from simplices}
A delta complex is a collection of simplices glued along the faces.

\begin{definition}
	Faces of a simplex followed by examples.
\end{definition}

\begin{definition}
	Definition of a delta complex followed by several examples.
\end{definition}

Talk about spheres, tori and projective spaces in 3 dimensions.

\begin{ques}
	When is a delta complex a manifold?
\end{ques}

Examples of manifolds and non-manifolds.





\end{document}
