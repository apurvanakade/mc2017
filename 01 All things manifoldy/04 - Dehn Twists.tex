\iffalse
It's unfair to say that mathematicians aren't real doctors, we perform surgeries all the time. In this class we'll introduce the notion of a topological manifold via simplicial (delta) complexes. Spend a day or two doing examples and go over several notions like orientation, cobordism and of course surgery.

Keywords: simplicial complex, manifold, orientation, cobordism, surgery

Type: Lecture
Homework: Recommended
Prereqs: None
\fi




\documentclass{article}
\usepackage{amsmath, amsthm}
\usepackage{amssymb}
\usepackage{mathtools}
\usepackage[all,cmtip]{xy}
\usepackage{color}


\setcounter{tocdepth}{4}

\renewenvironment{proof}{ {\bfseries Proof:}}{\qed}

\newtheoremstyle{mytheorem}%                % Name
{}%                                     % Space above
{}%                                     % Space below
{\itshape}%                                     % Body font
{0pt}%\parindent}%                                     % Indent amount
{\bfseries}%                            % Theorem head font
{.}%                                    % Punctuation after theorem head
{ }%                                    % Space after theorem head, ' ', or \newline
{}%                                     % Theorem head spec (can be left empty, meaning `normal')

\theoremstyle{mytheorem}
\newtheorem{thm}{Theorem}[section]
\newtheorem{proposition}[thm]{Proposition}
\newtheorem{lemma}[thm]{Lemma}
\newtheorem{corollary}[thm]{Corollary}


\newtheoremstyle{mydefinition}%                % Name
{}%                                     % Space above
{}%                                     % Space below
{}%                                     % Body font
{0pt}%\parindent}%                                     % Indent amount
{\bfseries}%                            % Theorem head font
{.}%                                    % Punctuation after theorem head
{ }%                                    % Space after theorem head, ' ', or \newline
{}%                                     % Theorem head spec (can be left empty, meaning `normal')

\theoremstyle{mydefinition}
\newtheorem{definition}[thm]{Definition}
\newtheorem{example}[thm]{Example}
\newtheorem{exercise}[thm]{Exercise}
\newtheorem{remark}[thm]{Remark}
%\newtheorem{ques}[thm]{Q.}
\newtheorem*{ques}{Question}
%\newtheorem{ans}[thm]{Ans.}
\newtheorem*{ans}{Ans}



\numberwithin{equation}{section}

%Real numbers, complex numbers, etc.
\newcommand{\R}{\mathbb{R}}
\newcommand{\C}{\mathbb{C}}
\newcommand{\Z}{\mathbb{Z}}
\newcommand{\Q}{\mathbb{Q}}
\renewcommand{\P}{\mathbb{P}}

%How does latex not have these?
\DeclareMathOperator{\Ad}{Ad}
\DeclareMathOperator{\ad}{ad}
\DeclareMathOperator{\tr}{tr}
\DeclareMathOperator{\Tr}{Tr}
\DeclareMathOperator{\Hom}{Hom}
\DeclareMathOperator{\Spec}{Spec}
\DeclareMathOperator{\im}{im}
\DeclareMathOperator{\rank}{rank}
\DeclareMathOperator{\Exists}{\exists}
\DeclareMathOperator{\Forall}{\forall}

\DeclareMathOperator*{\colim}{colim}
\DeclareMathOperator*{\holim}{holim}
\DeclareMathOperator*{\hocolim}{hocolim}


%fractions and inner product
\newcommand{\pr}[2][\:]{\frac{\partial #1}{\partial #2}}
\newcommand{\innerp}[2]{\langle #1, #2 \rangle}

\newcommand*\conj[1]{\overline{#1}}
\newcommand*\norm[1]{\lVert #1 \rVert}

\renewcommand{\figurename}{Fig.}
\usepackage{float}
\usepackage{wrapfig}

\usepackage{enumitem}
\setlist[enumerate]{itemsep=0mm}
\usepackage{geometry}
\geometry{
	a4paper,
	total={170mm,257mm},
	left=20mm,
	top=20mm
}


\usepackage{fancyhdr}
\pagestyle{fancy}
\lhead{\scshape Apurva Nakade}
%\rhead{\scshape Mathcamp 2017}
\renewcommand*{\thepage}{\small\arabic{page}}

\rhead{\scshape Mathcamp 2017 : All things Manifoldy}
\begin{document}
\title{Mapping class groups}
\author{Apurva Nakade}
\thispagestyle{fancy}
\maketitle



\section{Automorphisms}

\begin{definition}
	An \textbf{automorphism} or a \textbf{self-homeomorphism} of a manifold $X$ is a map $f:X \rightarrow X$ which is a homeomorphism. The set of automorphisms forms a group under composition, denoted ${\mathrm{Homeo}(X)}$.
\end{definition}

This group is usually too big to get a good handle on, so instead we study automorphisms up to deformations i.e. we consider two automorphisms to be the same if one automorphism can be continuously deformed into another.

\begin{definition}
	The group $\mathrm{Homeo}(X)/{deformations}$ is called the \textbf{mapping class group}, denoted $\mathrm{MCG}(X)$.
\end{definition}

\begin{example}\label{thm:MCG}
	Every automorphism of $\R^1$ is a strictly increasing or a strictly decreasing function $f:\R^1 \rightarrow \R^1$. It is possible to deform a strictly increasing function $f_1$ to another strictly increasing function $f_2$ via the path of maps $t.f_1 + (1-t).f_2$ for $t\in[0,1]$, similarly for decreasing functions. And composition of two decreasing functions is an increasing function. Together these imply that $\mathrm{MCG}(\R^1) \cong \Z/2$.
\end{example}

The main object of interest for us is the mapping class group of the torus $\mathrm{MCG}(T)$ i.e. automorphisms of the torus \emph{up to deformations}. Let us fix two non-parallel circles on the torus and call these the \textbf{principal circles}. While there are various choices for these all of which work we'll pick the simples ones and call them the \textbf{red} and the \textbf{blue} circles. See \textbf{Fig.1}.
%
% The first important observation to make is that \emph{up to deformations} an automorphism of a torus is completely determined by where the two principal circles are mapped.
%
%
% \begin{ques}
% 	Are there automorphisms of the torus which map the red lines as in \textbf{Fig.2}?
% \end{ques}


\section{Dehn Twists}
One way to construct non-trivial automorphisms of the torus is via Dehn twists.
\begin{definition}
	A \textbf{Dehn twist}, denoted $D$, is a special automorphism of the cylinder which twists the cylinder as in \textbf{Fig.2}.
\end{definition}

% The direction of the rotation is very important. If the cylinder is placed upright then we think of $D$ as fixing the bottom circle and twisting the cylinder by rotating the top circle counterclockwise. To avoid confusion we'll always indicate which way is \textbf{`up'} in the cylinder.

Dehn twist has the nice property that the two boundary circles are unchanged. We can use Dehn twists to create  non-trivial automorphisms of the torus by cutting out a cylinder, performing a Dehn twist, and glueing it back. This is an example of surgery on the torus! Dehn twists look even more interesting on gluing diagrams. See \textbf{Fig.3}.

We can perform Dehn twists on other cylinders sitting inside a torus which allows us to create more automorphisms of the torus. A theorem of Dehn-Lickorish  says that for genus $g$ surfaces the mapping class group is generated by a small set of Dehn twists. Dehn twists on punctured discs give rise to Braid groups establishing further connections between topology and group theory.


\section{Exercises}
\begin{exercise}
	Use the following exercises to show that the mapping class group of $S^1 = \{ (x,y): x^2 + y^2 = 1\}$ is $\Z/2$.
	\begin{enumerate}
		\item Show that every automorphism can be continuously deformed to one that fixes the point $(1,0)$.
		\item Find two automorphisms of $S^1$ which fix $(1,0)$ which cannot be deformed into each other via automorphisms.
		\item Show that $\mathrm{MCG}(S^1) \cong \Z/2$.
	\end{enumerate}
\end{exercise}

\begin{exercise}
	Find $\mathrm{MCG}(X)$ when $X$ is one of the following spaces
	\begin{enumerate}
		\item Two parallel lines in $\R^2$
		\item Two intersecting lines in $\R^2$
		\item Union of two intersecting circles
		\item The 2 dimensional unit disk $D^2 = \{ (x,y) \in \R^2 : x^2 + y^2 \le 1\}$ \\ This one is non-trivial. Read the wikipedia page on Alexander's trick.
	\end{enumerate}
\end{exercise}

\begin{exercise}
	Based on the above exercise what is the relationship between $\mathrm{MCG}(X \sqcup X)$ and $\mathrm{MCG}(X)$, where $X \sqcup X$ denotes the disjoint union of two copies of $X$.
\end{exercise}

\begin{exercise}
	Describe the homeomorphisms which are inverses of $D_R$ and $D_B$ in the mapping class group of the torus.
\end{exercise}

\begin{exercise}
	Perform a Dehn twist on the cylinder around the equator on $S^2$. What is the corresponding element in $\mathrm{MCG}(S^2)$?
\end{exercise}

\begin{exercise}
	\begin{enumerate}
		\item Verify that $D_B D_R D_B$ and $D_R D_B D_R$ are equal in $\mathrm{MCG}(T)$ by checking what they do to the principal circles.
		\item Consider the $2 \times 2$ matrices $M_B = \begin{bmatrix} 1 & 1 \\ 0 & 1 \end{bmatrix}$ and $M_R = \begin{bmatrix} 1 & 0 \\ -1 & 1 \end{bmatrix}$. Verify that $M_B M_R M_B = M_R M_B M_R$.
		\item Assuming that $\mathrm{MCG}(T)$ is generated by $D_R$ and $D_B$ show that this defines a homomorphism from $\mathrm{MCG}(T)$ to the group $SL_2(\Z)$ of $2 \times 2$ matrices with integer coefficients and determinant 1.
		\item Describe the action of the matrices $M_R$ and $M_B$ on the plane and relate it to the Dehn twists $D_R$ and $D_B$.
	\end{enumerate}
\end{exercise}

\begin{exercise}
	$D_B D_R D_B$ is NOT a reflection! In \textbf{Fig. 4} describe what Dehn twists do to the shaded regions and figure out what $D_B D_R D_B$ really is.
\end{exercise}



\end{document}
